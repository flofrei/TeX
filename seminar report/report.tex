\documentclass[12pt]{scrartcl}
\usepackage{graphicx}
%\usepackage[ngerman]{babel}
\usepackage{natbib}
\usepackage{amsmath}
\usepackage{amstext}
\usepackage{color}
\usepackage{braket}
\usepackage{lscape}
\usepackage{standalone}
\standaloneconfig{mode=buildnew}
\usepackage{tikz}
%\usepackage{pgfplots}
%\usepackage{xcolor}
\usetikzlibrary{positioning,shapes.geometric}
%\usetikzlibrary{trees}

\begin{document}
\author{Florian Frei}
\title{The Computational Bottleneck of Ab-initio Molecular Dynamics And Possible Solutions}

\maketitle

\newpage

\tableofcontents

\newpage

\section{Derivation of ab-initio molecular dynamics}
To understand where the computational bottleneck is coming from and why it is there one should start from the very beginning namely the Lagrangian and Hamilton formalism. These are essential that one does not forget the equivalence of these two views of the world. For the advanced reader it is recommended to skip these sections and start with derivation of classical molecular dynamics section \ref{sec:cmd}.
\subsection{Lagrangian of a system}
The Lagrangian is defined by
\begin{align}
\mathcal{L}(\mathbf{q},\dot{\mathbf{q}}) &= \mathcal{T}(\mathbf{q},\dot{\mathbf{q}}) - \mathcal{V}(\mathbf{q},\dot{\mathbf{q}})\label{def:lagrangian}\\
\intertext{with kinetic energy $\mathcal{T}$ and potential energy $\mathcal{V}$. In case of Born-Oppenheimer view the kinetic energy is dominated by nuclear motion and the potential energy is solely given by the electronic energy. Thus an educated guess requires that the Lagrangian is of the following form:}
\mathcal{L}^{BO}(\mathbf{R}_I,\dot{\mathbf{R}}_I) &= \frac{1}{2}\sum_{I}M_I\dot{\mathbf{R}}_I^2 - \mathcal{V}(\{\mathbf{R}_I\},D)\label{lagrangian:BO}\\
\intertext{where the potential is evaluated at the ground state $D$ which minimizes the electronic energy for a given nuclear configuration $\{R_I\}$. The equations of motion of the system is then given by the Euler-Lagrange equation:}
\frac{d}{dt}\frac{\partial \mathcal{L}}{\partial \dot{\mathbf{q}}} &=\frac{\partial \mathcal{L}}{\partial \mathbf{q}}\label{eq:euler-lagrange}
\end{align}

Inserting \ref{lagrangian:BO} into \ref{eq:euler-lagrange} and looking at each individual term results in:
\begin{align*}
\frac{\partial \mathcal{L}^{BO}}{\partial \mathbf{R}_I} &= - \frac{\partial \mathcal{V}(\{\mathbf{R}_I\},D)}{\partial \mathbf{R}_I}\\
\frac{\partial \mathcal{L}^{BO}}{\partial \dot{\mathbf{R}}_I} &= \frac{1}{2}\cdot 2 M_I \dot{\mathbf{R}}_I\\
\frac{d}{dt} \frac{\partial \mathcal{L}^{BO}}{\partial \dot{\mathbf{R}}_I} &= M_I \ddot{\mathbf{R}}_I\\
\implies M_I \ddot{\mathbf{R}}_I &= - \frac{\partial \mathcal{V}(\{\mathbf{R}_I\},D)}{\partial \mathbf{R}_I}\\
\iff M_I \ddot{\mathbf{R}}_I &= - \nabla_I \mathcal{V}(\{\mathbf{R}_I\},D)
\end{align*}
Therefore before integrating the nuclear motion one has to evaluate the potential energy which leads to search for the ground state $D$ for a given nuclear configuration.

\subsection{Changing to Hamiltonian function}
An equivalent formalism to the Lagrangian is the Hamiltonian which is in most cases easier to understand. The Legendre transformation can be used to change from Lagrangian to Hamiltonian formalism which is of the following form:

\begin{align}
\mathcal{H}(\mathbf{q}_j,\mathbf{p}_j,t) &= \sum_{i} \dot{\mathbf{q}}_i \frac{\partial \mathcal{L}}{\partial \dot{\mathbf{q}}_i} - \mathcal{L}(\mathbf{q}_j,\dot{\mathbf{q}}_j,t)
\intertext{where}
\frac{\partial \mathcal{L}}{\partial \dot{\mathbf{q}}_i} &= \mathbf{p}_i(\mathbf{q}_i,\dot{\mathbf{q}}_i,t)
\end{align}
is the momenta expressed with generalized coordinates $\mathbf{q}_i$ and velocities $\dot{\mathbf{q}}_i$. Therefore with the definition of the Born-Oppenheimer Lagrangian one can derive the Hamiltonian as:
\begin{align*}
\frac{\partial \mathcal{L}^{BO}}{\partial \dot{\mathbf{R}}_I} &= M_I \dot{\mathbf{R}}_I\\
\implies \dot{\mathbf{q}}_i \frac{\partial \mathcal{L}}{\partial \dot{\mathbf{q}}_i} &= \dot{\mathbf{R}}_I \cdot M_I \dot{\mathbf{R}}_I \\
\implies \mathcal{H}^{BO}(\mathbf{q}_j,\mathbf{p}_j,t)&=\sum_{I} M_I \dot{\mathbf{R}}_I^2 - \mathcal{L}^{BO}\\
\implies \mathcal{H}^{BO}(\mathbf{q}_j,\mathbf{p}_j,t)&=\frac{1}{2} \sum_{I} M_I \dot{\mathbf{R}}_I^2+ \mathcal{V}(\{\mathbf{R}_I\},D)\\
\iff \mathcal{H} &= \mathcal{T} + \mathcal{V}
\end{align*}
And additionally with looking at the total differential of the Lagrangian one can easily derive Hamiltonian's equations of motion.
%TODO
\begin{align*}
d\mathcal{L} &= \sum_{i} \left(  \frac{\partial \mathcal{L}}{\partial \mathbf{q}_i}d\mathbf{q}_i + \frac{\partial \mathcal{L}}{\partial \dot{\mathbf{q}}_i}d\dot{\mathbf{q}}_i  \right) + \frac{\partial \mathcal{L}}{\partial t} dt\\
d\mathcal{L} &= \sum_{i} \left(  \frac{\partial \mathcal{L}}{\partial \mathbf{q}_i}d\mathbf{q}_i + \mathbf{p}_id\dot{\mathbf{q}}_i  \right) + \frac{\partial \mathcal{L}}{\partial t} dt\\
d(\mathbf{p}_i\dot{\mathbf{q}}_i) &= d\mathbf{p}_i\dot{\mathbf{q}}_i+ \mathbf{p}_id\dot{\mathbf{q}}_i\\
d\mathcal{L} &= \sum_{i} \left(  \frac{\partial \mathcal{L}}{\partial \mathbf{q}_i}d\mathbf{q}_i +d(\mathbf{p}_i\dot{\mathbf{q}}_i) -  d\mathbf{p}_i\dot{\mathbf{q}}_i \right) + \frac{\partial \mathcal{L}}{\partial t} dt\\
d(\sum_{i} \mathbf{p}_i\dot{\mathbf{q}}_i - \mathcal{L})&=\sum_{i}\left(-\frac{\partial \mathcal{L}}{\partial \mathbf{q}_i}d\mathbf{q}_i +d\mathbf{p}_i\dot{\mathbf{q}}_i  \right)-\frac{\partial \mathcal{L}}{\partial t} dt \\
d\mathcal{H}&=\sum_{i}\left(-\frac{\partial \mathcal{L}}{\partial \mathbf{q}_i}d\mathbf{q}_i +d\mathbf{p}_i\dot{\mathbf{q}}_i  \right)-\frac{\partial \mathcal{L}}{\partial t} dt \\
d\mathcal{H}&=\sum_{i} \left(  \frac{\partial \mathcal{H}}{\partial \mathbf{q}_i}d\mathbf{q}_i + \frac{\partial \mathcal{H}}{\partial \mathbf{p}_i}d\mathbf{p}_i  \right) + \frac{\partial \mathcal{H}}{\partial t} dt\\
\implies \frac{\partial \mathcal{H}}{\partial \mathbf{q}_i}&=-\frac{\partial \mathcal{L}}{\partial \mathbf{q}_i},\;
\frac{\partial \mathcal{H}}{\partial \mathbf{p}_i}=\dot{\mathbf{q}}_i,\;
\frac{\partial \mathcal{H}}{\partial t}=-\frac{\partial \mathcal{L}}{\partial t}
\end{align*}
In case when the Euler-Lagrange equation \ref{eq:euler-lagrange} holds, it can be rewritten as:
\begin{equation}
\frac{\partial \mathcal{L}}{\partial \mathbf{q}_i} = \frac{d}{dt}\mathbf{p}_i=\dot{\mathbf{p}}_i
\end{equation}
and therefore also:
\begin{equation}
\frac{\partial \mathcal{H}}{\partial \mathbf{q}_j}=-\dot{\mathbf{p}}_j,\;\frac{\partial \mathcal{H}}{\partial \mathbf{p}_j}=\dot{\mathbf{q}}_j,\;\frac{\partial \mathcal{H}}{\partial t}=-\frac{\partial \mathcal{L}}{\partial t}
\end{equation}
Thus if one has the Hamiltonian of the system the derivation for the equations of motions is straightforward. A possible and well-known framework to accomplish this in molecular dynamics is to use Schr\"odinger's equation with using the correspondence principle, explained in the next section.

\subsection{Derivation of Classical Molecular Dynamics}
\label{sec:cmd}
To derive classical molecular dynamics a good choice is to start with non-relativistic quantum mechanics formalized via the time-dependent Schr\"odinger equation:
\begin{equation}
\mathrm{i} \hbar \frac{\partial }{\partial t} \Psi = \hat{\mathcal{H}} \Psi
\label{eq:tdse}
\end{equation}
, where the correspondence-principle was used to rewrite the energy equation
\begin{equation}
E=\frac{\mathbf{p}^2}{2m}+V(\mathbf{r},t)=\mathcal{T}(\mathbf{p})+\mathcal{V}(\mathbf{r},t)=\mathcal{H}
\end{equation} 
by replacing the non-relativistic quantities $E$, $\mathbf{p}$ and $\mathbf{r}$ with their corresponding operators
\begin{align*}
E \rightarrow \hat{E} &=\mathrm{i}\hbar \frac{\partial }{\partial t}\\
\mathbf{p} \rightarrow \hat{\mathbf{p}} &=-\mathrm{i}\hbar \nabla\\
\mathbf{r} \rightarrow \hat{\mathbf{r}}  &= \mathbf{r}\\
\mathcal{H} \rightarrow \hat{H} &=-\mathrm{i}\hbar \nabla^2 + \hat{V}(\mathbf{r},t))
\end{align*}
For this discussion $\Psi=\Psi(\{\mathbf{r}_i\},\{\mathbf{R}_I \};t)$ is a function depending on $N$ electrons with coordinates $\{\mathbf{r}_i\}_{i=1}^{N}$ and $M$ nuclei with coordinates $\{\mathbf{R}_I\}_{I=1}^{M}$ with a standard Hamiltonian constructed with the Coulomb potential:
\begin{align}
\hat{H} &=\hat{T}_n + \hat{T}_e + \hat{V}_{n-e}\label{def:hamiltonian}\\
\hat{T}_n &=-\sum_{I}\frac{\hbar^2}{2M_I} \nabla^2_I\label{def:nuclearkinetic}\\
\hat{T}_e &=-\sum_{i}\frac{\hbar^2}{2m_e} \nabla^2_i\label{def:electronickinetic}\\
\hat{V}_{n-e}&=\sum_{i<j}\frac{e^2}{|\mathbf{r}_i-\mathbf{r}_j|} - \sum_{I,i}\frac{e^2Z_I}{|\mathbf{R}_I-\mathbf{r}_i|} + \sum_{I<J}\frac{e^2Z_IZ_J}{|\mathbf{R}_I-\mathbf{R}_J|}\label{def:coulomboperator}\\
\hat{H}_{el}&=\hat{H}-\hat{T}_n=\hat{T}_e + \hat{V}_{n-e}\label{def:electronichamilton}
\end{align}

In case that one assumes that the time coordinate can be separated from the spatial coordinates one can use Louis de Broglies ansatz for standing waves
\begin{equation}
\Psi(\mathbf{r},t)=A\cdot \exp(-\frac{\mathrm{i}}{\hbar}(Et-\mathbf{r}\cdot\mathbf{p}))
\label{ansatz:louisdebroglieswave}
\end{equation}
to transform
\begin{align*}
\mathrm{i} \hbar \frac{\partial }{\partial t} (A\cdot \exp(-\frac{\mathrm{i}}{\hbar}(Et-\mathbf{r}\cdot\mathbf{p})) )&=\hat{H}\Psi(\mathbf{r},t)\\
\mathrm{i} \hbar  A\cdot \exp(-\frac{\mathrm{i}}{\hbar}(Et-\mathbf{r}\cdot\mathbf{p})) \frac{\partial }{\partial t}(-\frac{\mathrm{i}}{\hbar}(Et-\mathbf{r}\cdot\mathbf{p})) &=\hat{H}\Psi(\mathbf{r},t)\\
\mathrm{i} \hbar\cdot -\frac{\mathrm{i}}{\hbar} \cdot E \cdot A\cdot \exp(-\frac{\mathrm{i}}{\hbar}(Et-\mathbf{r}\cdot\mathbf{p}))  &=\hat{H}\Psi(\mathbf{r},t)\\
\mathrm{i} \hbar\cdot -\frac{\mathrm{i}}{\hbar} \cdot E \cdot \Psi(\mathbf{r},t)&=\hat{H}\Psi(\mathbf{r},t)\\
 E \Psi(\mathbf{r},t)&=\hat{H}\Psi(\mathbf{r},t)\\
\end{align*}
into the time independent Schr\"odinger equation
\begin{equation}
\hat{H}\Psi(\mathbf{r},t)=E \Psi(\mathbf{r},t)
\label{eq:tiseclassic}
\end{equation}
where this is equivalent to an eigenvalue problem
\begin{equation}
\mathcal{M} \mathbf{v} = \lambda \mathbf{v}
\label{eigenvalueproblem}
\end{equation}
with matrix $\mathcal{M}$, eigenvector $\mathbf{v}$ and eigenvalue $\lambda$. Hence it is of importance to choose the right basis such that the Hamiltonian can expressed as a matrix.

\section{Solving Schr\"odinger equation with two different Ansatz}
\subsection{Ansatz for TDSCF}
For solving the time-dependent Schr\"odinger equation \ref{eq:tdse} one has to use an ansatz. This ansatz is assumed to be true and in most cases an educated guess is made. The simplest case is to assume that electronic and nuclear part are separable in product form in the following way:
\begin{equation}
\Psi( \{ \mathbf{r}_i \}, \{ \mathbf{R}_I \};t) \approx \chi( \{ \mathbf{R}_I \};t) \prod_{k} \Phi_k( \{ \mathbf{r}_i \} ;t)
\label{ansatz:tdscf}
\end{equation}
Worth noting is that this ansatz yields the time-dependent self-consistent field method derived as early as 1930 by Dirac \cite{dirac1930note}. This derivation will be shown for the purpose of illustrating where the interdependence of the final equations are coming from. For further reference on this derivation one can look in \cite{makri1987time} for a more complex Hamiltonian function. First step is to insert the ansatz \ref{ansatz:tdscf} into the time-dependent Schr\"odinger equation \ref{eq:tdse}.
\begin{equation}
\mathrm{i}\hbar \frac{\partial}{\partial t} \left(\chi( \{ \mathbf{R}_I \};t) \prod_{k} \Phi_k( \{ \mathbf{r}_i \} ;t)\right) = \left( \hat{T}_n + \hat{T}_e + \hat{V}_{n-e} \right)\chi( \{ \mathbf{R}_I \};t) \prod_{k} \Phi_k( \{ \mathbf{r}_i \} ;t)
\label{eq:tdsewithansatz}
\end{equation}
This equation can be separated into two parts. One is for describing the nuclear part using a projection on to electronic coordinates and vice versa. Derivation for the nuclear part involve multiplying $\prod_{k} \Bra{\Phi_k}$ from the left which is equivalent to the projection. Hence the equation \ref{eq:tdsewithansatz} can be expanded and reformulated as:
\begin{align*}
&\prod_{k} \Bra{\Phi_k} \mathrm{i}\hbar \frac{\partial \chi}{\partial t} \prod_{k} \Phi_k + \prod_{k} \Bra{\Phi_k} \mathrm{i}\hbar \chi \sum_{j}\prod_{k,k \neq j} \frac{\partial\Phi_j}{\partial t}\Phi_k \\&= \prod_{k} \Bra{\Phi_k} \hat{T}_n \chi \prod_{k} \Phi_k + \prod_{k} \Bra{\Phi_k} \hat{T}_e \chi \prod_{k} \Phi_k + \prod_{k} \Bra{\Phi_k} \hat{V}_{n-e} \chi \prod_{k} \Phi_k
\end{align*}
The individual terms can be all simplified because of the property of the orbital functions. First of all,
\begin{align*}
\prod_{k} \Bra{\Phi_k} \prod_{j}\Phi_j = \prod_{k,j}\Braket{\Phi_k | \Phi_j}=\delta_{kj}
\intertext{, can be used since orbitals have to be orthogonal to each other, hence the only interesting case is $k=j$. Note that we used the Kronecker delta on the right side. Therefore also the following holds:}
\prod_{k} \Bra{\Phi_k} \sum_{j}\prod_{k,k \neq j} \frac{\partial\Phi_j}{\partial t}\Phi_k = \sum_{j} \Braket{\Phi_j|\frac{\partial\Phi_j}{\partial t}}
\end{align*}
One can easily convince oneself that the all terms cancel each other except the $j$ term because of the time derivative. Now since this projection is independent of the $\chi$ functions they can be extracted to the right which is shown next:
\begin{align*}
\mathrm{i}\hbar \frac{\partial \chi}{\partial t} +\mathrm{i}\hbar \sum_{j} \Braket{\Phi_j|\frac{\partial\Phi_j}{\partial t}}\chi &= \hat{T}_n\chi+ \sum_{i} \Bra{\Phi_i} \hat{T}_e(i)\Ket{\Phi_i} \chi + \sum_{i} \Bra{\Phi_i} \hat{V}_{n-e}(i)\Ket{\Phi_i} \chi \\
\intertext{This is valid because kinetic operator of electrons and the Coulomb term is a summation over all coordinates $\{\mathbf{r}_i \}$}
\mathrm{i}\hbar \frac{\partial \chi}{\partial t} + \mathrm{i}\hbar\sum_{j} \Braket{\Phi_j|\frac{\partial\Phi_j}{\partial t}}\chi &= \hat{T}_n\chi+ \sum_{i} \Bra{\Phi_i} \hat{T}_e(i)+\hat{V}_{n-e}(i)\Ket{\Phi_i} \chi\\
\intertext{These two operators can be combined to the $\hat{H}_{el}$ where in Bra and Ket are all electronical coordinates. Also the summation and integral can freely be exchanged.}
\mathrm{i}\hbar \frac{\partial \chi}{\partial t} + \mathrm{i}\hbar\sum_{j} \Braket{\Phi_j|\frac{\partial\Phi_j}{\partial t}}\chi &= \hat{T}_n\chi+  \Bra{\Phi_i} \hat{H}_{el}\Ket{\Phi_i} \chi\\
\end{align*}
This equation can be further simplified with the introduction of a phase factor into the $\chi$ function. The role of this factor is to cancel the term including the time derivative of the electronic orbital. This can be achieved with the following substitution:
\begin{equation}
\chi \rightarrow \chi \cdot \exp\left( \frac{\mathrm{i}}{\hbar} \int_{t0}^{t}dt'\sum_{i}\mathrm{i}\hbar\Braket{\Phi_i|\dot{\Phi}_i} \right)
\label{phasefactorsinglechi}
\end{equation}
This can be done since the term in the exponential function is only dependent on the electronic configuration and not on the nuclear part. Therefore only the $\chi$ time derivative with the product rule will yield a new term, namely
\begin{align*}
\frac{\partial \chi \exp\left( \frac{\mathrm{i}}{\hbar} \int_{t0}^{t}dt'\sum_{i}\mathrm{i}\hbar\Braket{\Phi_i|\dot{\Phi}_i} \right)}{\partial t} = \frac{\partial \chi}{\partial t}\exp\left(...\right) + \chi\exp\left(...\right) \cdot \frac{\mathrm{i}}{\hbar} \cdot \sum_{i}\mathrm{i}\hbar\Braket{\Phi_i|\dot{\Phi}_i}\\
=\frac{\partial \chi}{\partial t}\exp\left(...\right) - \chi\exp\left(...\right) \sum_{i}\Braket{\Phi_i|\dot{\Phi}_i}
\end{align*}
Which will be inserted into the last equation and the exponential can be canceled on each side yielding:
\begin{equation}
\mathrm{i}\hbar \frac{\partial \chi}{\partial t} = \hat{T}_n\chi+  \Bra{\Phi_i} \hat{H}_{el}\Ket{\Phi_i} \chi
\label{eqfornuclearpart}
\end{equation}
Now one can look at the projection on to the $\chi$ function which is equivalent to multiply equation \ref{eq:tdsewithansatz} from the left with $\Bra{\chi}$ expanding and reformulating result to:
\begin{align*}
&\Bra{\chi}\mathrm{i}\hbar \frac{\partial \chi}{\partial t} \prod_{k} \Phi_k + \Bra{\chi} \mathrm{i}\hbar \chi \sum_{j}\prod_{k,k \neq j} \frac{\partial\Phi_j}{\partial t}\Phi_k \\ &= \Bra{\chi} \hat{T}_n \chi \prod_{k} \Phi_k + \Bra{\chi} \hat{T}_e \chi \prod_{k} \Phi_k + \Bra{\chi} \hat{V}_{n-e} \chi \prod_{k} \Phi_k\\
\intertext{Also the nuclear functions are orthogonal hence $\Braket{\chi | \chi}=1$. Furthermore the electronic part is independent from the nuclear part which can be separated into:}
&\mathrm{i}\hbar \Braket{\chi|\frac{\partial \chi}{\partial t}} \prod_{k} \Phi_k + \mathrm{i}\hbar \sum_{j}\prod_{k,k \neq j} \frac{\partial\Phi_j}{\partial t}\Phi_k \\&= \Bra{\chi}\hat{T}_n\Ket{\chi} \prod_{k} \Phi_k + \hat{T}_e \prod_{k} \Phi_k + \Bra{\chi} \hat{V}_{n-e} \Ket{\chi} \prod_{k} \Phi_k
\end{align*}
To not lose the total overview let us first look at a single $\phi$ function before generalizing to the whole product. With a single function the equation reduces to:
\begin{align*}
\mathrm{i}\hbar \Braket{\chi|\frac{\partial \chi}{\partial t}} \Phi + \mathrm{i}\hbar \frac{\partial\Phi}{\partial t} = \Bra{\chi}\hat{T}_n\Ket{\chi} \Phi + \hat{T}_e \Phi + \Bra{\chi} \hat{V}_{n-e} \Ket{\chi} \Phi
\end{align*}
Here one can also introduce a phase factor for the $\Phi$ function to further reduce the equation. Note however that the phase factor only include terms depending on the nuclear function. With $\Phi$ becoming,
\begin{equation}
\Phi \rightarrow \Phi \cdot \exp\left(\frac{\mathrm{i}}{\hbar} \int_{t0}^{t}dt' \mathrm{i}\hbar \Braket{\chi|\frac{\partial \chi}{\partial t}} - \Bra{\chi}\hat{T}_n\Ket{\chi}   \right)
\label{phasefactorsinglephi}
\end{equation}
, the partial time derivative therefore is:
\begin{align*}
\frac{\partial \Phi\exp(...)}{\partial t} = \frac{\partial \Phi}{\partial t}+\Phi\exp(...)\cdot \frac{\mathrm{i}}{\hbar} \cdot (\mathrm{i}\hbar \Braket{\chi|\frac{\partial \chi}{\partial t}} - \Bra{\chi}\hat{T}_n\Ket{\chi})
\intertext{and hence with a prefactor:}
\mathrm{i}\hbar \Braket{\chi|\frac{\partial \chi}{\partial t}} \Phi\exp(...) + \mathrm{i}\hbar \frac{\partial \Phi}{\partial t}+\Phi\exp(...)\cdot (-\mathrm{i}\hbar\Braket{\chi|\frac{\partial \chi}{\partial t}} +\Bra{\chi}\hat{T}_n\Ket{\chi}) = \\ \Bra{\chi}\hat{T}_n\Ket{\chi} \Phi\exp(...) + \hat{T}_e \Phi\exp(...) + \Bra{\chi} \hat{V}_{n-e} \Ket{\chi} \Phi\exp(...)
\end{align*}
And thus the exponential function cancels on each side and two terms were reduced to the final form of:
\begin{equation}
\mathrm{i}\hbar \frac{\partial \Phi}{\partial t} =\hat{T}_e \Phi + \Bra{\chi} \hat{V}_{n-e} \Ket{\chi} \Phi
\label{eqforelectricpart}
\end{equation}
The phase factors of equation \ref{eqfornuclearpart} and \ref{eqforelectricpart} can be combined as follows:
\begin{align*}
&\chi\Phi \exp\left(\frac{\mathrm{i}}{\hbar} \int_{t0}^{t}dt' \mathrm{i}\hbar \Braket{\chi|\dot{\chi}} - \Bra{\chi}\hat{T}_n\Ket{\chi}   \right) \exp\left( \frac{\mathrm{i}}{\hbar} \int_{t0}^{t}dt'\mathrm{i}\hbar\Braket{\Phi|\dot{\Phi}} \right)\\ 
&=\chi \Phi \exp\left(\frac{\mathrm{i}}{\hbar} \int_{t0}^{t}dt' \mathrm{i}\hbar\Braket{\chi|\dot{\chi}} - \Bra{\chi}\hat{T}_n\Ket{\chi}+\mathrm{i}\hbar\Braket{\Phi|\dot{\Phi}}\right)
\intertext{were we can expand terms with $\Braket{\chi|\chi}$ and $\Braket{\Phi|\Phi}$:}
&=\chi \Phi \exp\left(\frac{\mathrm{i}}{\hbar} \int_{t0}^{t}dt' \mathrm{i}\hbar\Braket{\chi|\dot{\chi}}\Braket{\Phi|\Phi} - \Braket{\Phi|\Phi}\Bra{\chi}\hat{T}_n\Ket{\chi}+\mathrm{i}\hbar\Braket{\chi|\chi}\Braket{\Phi|\dot{\Phi}}\right)
\intertext{which can be further collapsed into:}
&=\chi \Phi \exp\left(\frac{\mathrm{i}}{\hbar} \int_{t0}^{t}dt' \Bra{\Phi\chi} \mathrm{i}\hbar (\phi\frac{\partial \chi}{\partial t}+\chi\frac{\partial \Phi}{\partial t})\Ket{} - \Bra{\Phi\chi}\hat{T}_n\Ket{\Phi\chi} \right)\\
&=\chi \Phi \exp\left(\frac{\mathrm{i}}{\hbar} \int_{t0}^{t}dt' \Bra{\Phi\chi} \mathrm{i}\hbar \frac{\partial}{\partial t}(\Phi\chi) - \Bra{\Phi\chi}\hat{T}_n\Ket{\Phi\chi} \right)
\intertext{Insert again time dependent Schr\"odinger eq.}
&=\chi \Phi \exp\left(\frac{\mathrm{i}}{\hbar} \int_{t0}^{t}dt' \Bra{\Phi\chi} \hat{H}\Ket{\Phi\chi} - \Bra{\Phi\chi}\hat{T}_n\Ket{\Phi\chi} \right)\\
&=\chi \Phi \exp\left(\frac{\mathrm{i}}{\hbar} \int_{t0}^{t}dt' \Bra{\Phi\chi} \hat{H}_{el}\Ket{\Phi\chi} \right)\\
\end{align*}
This means the most compact form for an ansatz with only one electronic function and phasefactor is:
\begin{equation}
\Psi( \{ \mathbf{r}_i \}, \{ \mathbf{R}_I \};t) \approx \chi( \{ \mathbf{R}_I \};t) \Phi( \{ \mathbf{r}_i \} ;t)\cdot \exp\left(\frac{\mathrm{i}}{\hbar} \int_{t0}^{t}dt' \Bra{\Phi\chi} \hat{H}_{el}\Ket{\Phi\chi} \right)
\label{eq:tdsewithansatz2}
\end{equation}
Now lets return to a case with more than one electronic function. The equation takes the following form:
\begin{align*}
&\mathrm{i}\hbar \Braket{\chi|\frac{\partial \chi}{\partial t}} \Phi_1\Phi_2 + \mathrm{i}\hbar \frac{\partial\Phi_1}{\partial t}\Phi_2+\mathrm{i}\hbar\frac{\partial\Phi_2}{\partial t}\Phi_1\\
&= \Bra{\chi}\hat{T}_n\Ket{\chi} \Phi_1\Phi_2 + \hat{T}_e \Phi_1\Phi_2 + \Bra{\chi} \hat{V}_{n-e} \Ket{\chi} \Phi_1\Phi_2
\intertext{Now consider a projection on $\Phi_2$.}
&\Bra{\Phi_2}\mathrm{i}\hbar \Braket{\chi|\frac{\partial \chi}{\partial t}} \Phi_1\Phi_2 + \Bra{\Phi_2}\mathrm{i}\hbar \frac{\partial\Phi_1}{\partial t}\Phi_2+\Bra{\Phi_2}\mathrm{i}\hbar\frac{\partial\Phi_2}{\partial t}\Phi_1\\
&= \Bra{\Phi_2} \Bra{\chi}\hat{T}_n\Ket{\chi} \Phi_1\Phi_2 + \Bra{\Phi_2} \hat{T}_e \Phi_1\Phi_2 + \Bra{\Phi_2} \Bra{\chi} \hat{V}_{n-e} \Ket{\chi} \Phi_1\Phi_2
\intertext{With $\Braket{\Phi_2|\Phi_2}=1$ this gives us:}
&\mathrm{i}\hbar \Braket{\chi|\frac{\partial \chi}{\partial t}} \Phi_1 + \mathrm{i}\hbar \frac{\partial\Phi_1}{\partial t}+\mathrm{i}\hbar\Braket{\Phi_2|\frac{\partial\Phi_2}{\partial t}}\Phi_1\\
&= \Bra{\chi}\hat{T}_n\Ket{\chi} \Phi_1 + \Bra{\Phi_2} \hat{T}_e\Ket{\Phi_2} \Phi_1 + \Bra{\chi\Phi_2} \hat{V}_{n-e} \Ket{\chi\Phi_2} \Phi_1
\intertext{Introducting again a phasefactor to $\Phi_1$ function:}
&\Phi_1 \rightarrow \Phi_1 \cdot \exp\left(\frac{\mathrm{i}}{\hbar} \int_{t0}^{t}dt' \mathrm{i}\hbar \Braket{\chi|\dot{\chi}} +\mathrm{i}\hbar\Braket{\Phi_2|\dot{\Phi_2}}  - \Bra{\chi}\hat{T}_n\Ket{\chi}   \right)\\
&\mathrm{i}\hbar \frac{\partial\Phi_1}{\partial t}= \Bra{\Phi_2} \hat{T}_e\Ket{\Phi_2} \Phi_1 + \Bra{\chi\Phi_2} \hat{V}_{n-e} \Ket{\chi\Phi_2} \Phi_1
\intertext{And vice versa:}
&\Phi_2 \rightarrow \Phi_2 \cdot \exp\left(\frac{\mathrm{i}}{\hbar} \int_{t0}^{t}dt' \mathrm{i}\hbar \Braket{\chi|\dot{\chi}} +\mathrm{i}\hbar\Braket{\Phi_1|\dot{\Phi_1}}  - \Bra{\chi}\hat{T}_n\Ket{\chi}   \right)\\
&\mathrm{i}\hbar \frac{\partial\Phi_2}{\partial t}= \Bra{\Phi_1} \hat{T}_e\Ket{\Phi_1} \Phi_2 + \Bra{\chi\Phi_1} \hat{V}_{n-e} \Ket{\chi\Phi_1} \Phi_2
\intertext{Recombining all phasefactors:}
&\chi \Phi_1 \Phi_2 \exp( \frac{\mathrm{i}}{\hbar} \int_{t0}^{t}dt'\sum_{i}^{2}\mathrm{i}\hbar\Braket{\Phi_i|\dot{\Phi}_i} \\
&+\mathrm{i}\hbar\Braket{\chi|\dot{\chi}} +\mathrm{i}\hbar\Braket{\Phi_1|\dot{\Phi_1}}  - \Bra{\chi}\hat{T}_n\Ket{\chi}\\
&+\mathrm{i}\hbar \Braket{\chi|\dot{\chi}} +\mathrm{i}\hbar\Braket{\Phi_2|\dot{\Phi_2}}  - \Bra{\chi}\hat{T}_n\Ket{\chi} )
\intertext{With analogous reformulation one can derive:}
&\chi \Phi_1 \Phi_2 \exp\left( \frac{\mathrm{i}}{\hbar} 2 \cdot \int_{t0}^{t}dt' \Bra{\chi\Phi_1\Phi_2}\hat{H}_{el}\Ket{\chi\Phi_1\Phi_2} \right)
\end{align*}
Now one can derive easily the formulas for the whole product:
\begin{align}
&\mathrm{i}\hbar \frac{\partial \chi}{\partial t} = \hat{T}_n\chi+  \Bra{\Phi_i} \hat{H}_{el}\Ket{\Phi_i} \chi, \quad  i=1...N \label{eq:tdscfnuclear}\\
&\mathrm{i}\hbar \frac{\partial\Phi_k}{\partial t}= \Bra{\Phi_{k'}} \hat{T}_e\Ket{\Phi_{k'}} \Phi_k + \Bra{\chi\Phi_{k'}} \hat{V}_{n-e} \Ket{\chi\Phi_{k'}} \Phi_k, \quad k,k'=1...N \wedge k'\neq k\label{eq:tdscfelectronic}\\
&\Psi \approx \chi \prod_{k}\Phi_k \cdot \exp\left( \frac{\mathrm{i}}{\hbar} N \cdot \int_{t0}^{t}dt' \Bra{\chi\Phi_i}\hat{H}_{el}\Ket{\chi\Phi_i} \right),\quad  i=1...N \label{ansatz:tdscfwithphase}
\end{align}
Here one can easily see the interdependence of electronic and nuclear coordinates for the SCF method. In each timestep the orbital functions have to iteratively optimized until all equations are fulfilled. This is the computational bottleneck for choosing this method. Now let us have a look at an alternative method which will be derived from the Born-Oppenheimer approximation.
%TODO

\subsection{Ansatz for BOMD}
As already briefly mentioned the big difference for this approximation is that we chose the electronic part time independent. Therefore one has to find a solution for the eigenvalue problem \ref{eigenvalueproblem}. This is a valid view because the motion of the nuclei dominate all motion of electrons because of significant mass differences. Meaning one has to integrate the nuclear motion on a energy surface which minimizes the electronic energy. Therefore one can introduce the electronic time independent Schr\"odinger equation where the nuclei are stationary parameters.
\begin{equation}
\hat{H}(\{\mathbf{r}_i\},\{\mathbf{R}_I\}) \Psi(\{\mathbf{r}_i\},\{\mathbf{R}_I\}) = E\Psi(\{\mathbf{r}_i\},\{\mathbf{R}_I\})
\label{eq:tise}
\end{equation}
Using the eigenfunctions $\Psi_k$ with all possible configurations as a basis for the total wavefunction.
Introducing a complete basis for $\Psi_K$ which uniquely describe the wavefunctions:
\begin{align}
&\Psi( \{ \mathbf{r}_i \}, \{ \mathbf{R}_I \};t) = \sum_{m=0}^{\infty} \chi_{k,m}( \{ \mathbf{R}_I \};t)\Psi_{el,m}( \{ \mathbf{r}_i \},\{ \mathbf{R}_I \})\label{basis:exact}
\intertext{Whereas each basis function has to fulfill the electronic Schr\"odinger equation.}
&\hat{H}_{el}\Psi_{el,m}^{\{\mathbf{R}_I\}}(\{\mathbf{r}_i\}) = E_{el,m} \Psi_{el,m}^{\{\mathbf{R}_I\}}(\{\mathbf{r}_i\})\label{eq:tiese}
\end{align}
%TODO
In the Born-Oppenheimer approximation only one leading term will be used:
\begin{equation}
\Psi(\{\mathbf{r}_i\},\{\mathbf{R}_I\}) \approx \chi_{k,m}(\{ \mathbf{R}_I\})\cdot \Psi_{el,m}^{\{ \mathbf{R}_I\}}(\{ \mathbf{r}_i\})
\label{ansatz:bo}
\end{equation}
Inserting ansatz \ref{ansatz:bo} into time independent Schr\"odinger equation \ref{eq:tise} gives:
\begin{align*}
\hat{H}(\chi_{k,m}\cdot \Psi_{el,m}) &= E_k\chi_{k,m}\Psi_{el,m}\\
\hat{T}_n\chi_{k,m}\Psi_{el,m}+\hat{H}_{el}\chi_{k,m}\Psi_{el,m} &= E_k\chi_{k,m}\Psi_{el,m}
\intertext{$\hat{H}_{el}$ has no influence on $\chi_{k,m}$ and vice versa $\hat{T}_n$ has no influence on $\Psi_{el,m}$. Therefore a projection on to $\Psi_{el,m}$ yields:}
\Bra{\Psi_{el,m}} \hat{T}_n\chi_{k,m}\Psi_{el,m}+\Bra{\Psi_{el,m}}\hat{H}_{el}\chi_{k,m}\Psi_{el,m} &= \Bra{\Psi_{el,m}}E_k\chi_{k,m}\Psi_{el,m}\\
\hat{T}_n\chi_{k,m}\Braket{\Psi_{el,m}|\Psi_{el,m}}+\Bra{\Psi_{el,m}}\hat{H}_{el}(\{ \mathbf{r}_i\},\{ \mathbf{R}_I\})\Ket{\Psi_{el,m}}\chi_{k,m} &= E_k\chi_{k,m}\Braket{\Psi_{el,m}|\Psi_{el,m}}
\end{align*}
Where $\Braket{\Psi_{el,m}|\Psi_{el,m}}=\delta_{mm}=1$ and $\Bra{\Psi_{el,m}}\hat{H}_{el}(\{ \mathbf{r}_i\},\{ \mathbf{R}_I\})\Ket{\Psi_{el,m}}=E_{el,m}$ hence the final nuclear Schr\"odinger equation in BO-approximation is:
\begin{equation}
\left[ \hat{T}_n+E_{el,m}(\{\mathbf{R}_I\})\right]\chi_{k,m}=E_k\chi_{k,m}
\label{eq:nse}
\end{equation}
The structure of equation \ref{eq:nse} shows that this is again an eigenvalue problem but here for the nuclear orbital functions. Meaning that to solve this equation one has to first calculate the electronic energy for the given nuclear configuration $\{\mathbf{R}_I\}$ which is given by solving \ref{eq:tiese}. For completeness the general case where the exact representation \ref{basis:exact} is used will be derived next. Inserting \ref{basis:exact} into \ref{eq:tise} yields:
\begin{align*}
\hat{H} \sum_{m}\chi_{k,m}\Psi_{el,m}&=E_k\sum_{m}\chi_{k,m}\Psi_{el,m}
\intertext{Using again the definition \ref{def:electronichamilton}}
\hat{T}_n\sum_{m}\chi_{k,m}\Psi_{el,m} + \hat{H}_{el}\sum_{m}\chi_{k,m}\Psi_{el,m}&=E_k\sum_{m}\chi_{k,m}\Psi_{el,m}
\intertext{Project on to $\Bra{\Psi_{el,p}}$.}
\Bra{\Psi_{el,p}}\hat{T}_n\Ket{\sum_{m}\chi_{k,m}\Psi_{el,m}} + \Bra{\Psi_{el,p}}\hat{H}_{el}\Ket{\sum_{m}\chi_{k,m}\Psi_{el,m}}&=\Bra{\Psi_{el,p}}E_k\Ket{\sum_{m}\chi_{k,m}\Psi_{el,m}}
\intertext{Integral and sum can be exchanged.}
\sum_{m}\Bra{\Psi_{el,p}}\hat{T}_n\Ket{\chi_{k,m}\Psi_{el,m}} + \sum_{m}\Bra{\Psi_{el,p}}\hat{H}_{el}\Ket{\chi_{k,m}\Psi_{el,m}}&=\sum_{m}\Bra{\Psi_{el,p}}E_k\Ket{\chi_{k,m}\Psi_{el,m}}
\intertext{The electronic Hamiltonian and $E_k$ have no influence on $\chi$ functions.}
\sum_{m}\Bra{\Psi_{el,p}}\hat{T}_n\Ket{\chi_{k,m}\Psi_{el,m}} + \sum_{m}\Bra{\Psi_{el,p}}\hat{H}_{el}\Ket{\Psi_{el,m}}\chi_{k,m}&=\sum_{m}\Bra{\Psi_{el,p}}E_k\Ket{\Psi_{el,m}}\chi_{k,m}
\intertext{$E_k$ is a scalar.}
\sum_{m}\Bra{\Psi_{el,p}}\hat{T}_n\Ket{\chi_{k,m}\Psi_{el,m}} + \sum_{m}\Bra{\Psi_{el,p}}\hat{H}_{el}\Ket{\Psi_{el,m}}\chi_{k,m}&=E_k\sum_{m}\Braket{\Psi_{el,p}|\Psi_{el,m}}\chi_{k,m}
\intertext{Using Kronecker delta.}
\sum_{m}\Bra{\Psi_{el,p}}\hat{T}_n\Ket{\chi_{k,m}\Psi_{el,m}} + \sum_{m}\Bra{\Psi_{el,p}}\hat{H}_{el}\Ket{\Psi_{el,m}}\chi_{k,m}&=E_k\sum_{m}\delta_{pm}\chi_{k,m}
\intertext{The sum collapses on the right hand side except $p=m$.}
\sum_{m}\Bra{\Psi_{el,p}}\hat{T}_n\Ket{\chi_{k,m}\Psi_{el,m}} + \sum_{m}\Bra{\Psi_{el,p}}\hat{H}_{el}\Ket{\Psi_{el,m}}\chi_{k,m}&=E_k\chi_{k,m}
\intertext{Using electronic Schr\"odinger equation \ref{eq:tiese}.}
\sum_{m}\Bra{\Psi_{el,p}}\hat{T}_n\Ket{\chi_{k,m}\Psi_{el,m}} + \sum_{m}\Bra{\Psi_{el,p}}E_{el,m}\Ket{\Psi_{el,m}}\chi_{k,m}&=E_k\chi_{k,m}
\intertext{$E_{el,m}$ is again a scalar. With Kronecker delta the second sum collapses.}
\sum_{m}\Bra{\Psi_{el,p}}\hat{T}_n\Ket{\chi_{k,m}\Psi_{el,m}} + E_{el,m}\chi_{k,m}&=E_k\chi_{k,m}
\end{align*}
Looking at the definition \ref{def:nuclearkinetic} for $\hat{T}_n=-\sum_{I}\frac{1}{2M_I}\Delta_I$, the Laplace operator has to be evaluated.
\begin{align*}
\nabla_I^2(\chi_{k,m}\Psi_{el,m})=\nabla_I \cdot (\Psi_{el,m}\nabla_I\chi_{k,m}+\chi_{k,m}\nabla_I\Psi_{el,m}) \\
=\Psi_{el,m}\Delta_I\chi_{k,m}+2(\nabla_I\chi_{k,m})\cdot(\nabla_I\Psi_{el,m})+\chi_{k,m}\Delta_I\Psi_{el,m}
\end{align*}
Hence:
\begin{align*}
\Psi_{el,m}\Delta_I\chi_{k,m} &\rightarrow \sum_{m}\sum_{I}\Braket{\Psi_{el,p}|\Psi_{el,m}}\left(-\frac{1}{2M_I}\Delta_I \right)\chi_{k,m}\\
2(\nabla_I\chi_{k,m})\cdot(\nabla_I\Psi_{el,m})&\rightarrow \sum_{m}\sum_{I}2\Bra{\Psi_{el,p}}\nabla_I\Ket{\Psi_{el,m}}\left(-\frac{1}{2M_I}\nabla_I \right)\chi_{k,m}\\
\chi_{k,m}\Delta_I\Psi_{el,m} &\rightarrow \sum_{m}\sum_{I}\Bra{\Psi_{el,p}}\left(-\frac{1}{2M_I}\Delta_I \right)\Ket{\Psi_{el,m}}\chi_{k,m}\\
\end{align*}
The first term is again equivalent to $\hat{T}_n\chi_{k,m}$. Summarized the final equation is of the following form:
\begin{align*}
\hat{T}_n\chi_{k,m} +\sum_{m}\sum_{I}2\Bra{\Psi_{el,p}}\nabla_I\Ket{\Psi_{el,m}}\left(-\frac{1}{2M_I}\nabla_I \right)\chi_{k,m} +\\ \sum_{m}\sum_{I}\Bra{\Psi_{el,p}}\left(-\frac{1}{2M_I}\Delta_I \right)\Ket{\Psi_{el,m}}\chi_{k,m}
+ E_{el,m}\chi_{k,m}=E_k\chi_{k,m}
\end{align*}
The double sum is also called the non-adiabatic coupling term $\hat{C}_{pm}$.
\begin{equation}
\hat{C}_{pm} = \sum_{I}2\Bra{\Psi_{el,p}}\nabla_I\Ket{\Psi_{el,m}}\left(-\frac{1}{2M_I}\nabla_I\right)+\sum_I\Bra{\Psi_{el,p}} \left(-\frac{1}{2M_I}\Delta_I\right)\Ket{\Psi_{el,m}}
\end{equation}
And therefore the exact nuclear Schr\"odinger equation ca be written as:
\begin{equation}
\hat{T}_n\chi_{k,m} + \sum_{m}\hat{C}_{pm}\chi_{k,m}+E_{el,m}\chi_{k,m}=E_k\chi_{k,m}
\label{eq:entise}
\end{equation}
The $\hat{C}_{pm} $ terms describe the resonance between oscillation levels of neighboring electronic states. Approximations therefore make only sense when the electronic states in a system are not coupled. There are two famous approximations. First the adiabatic approximation:
\begin{align}
\hat{C}_{pm}=0 \;  \forall m \neq n \; & \rightarrow \;\hat{T}_n\chi_{k,m}+\hat{C}_{mm}\chi_{k,m}+E_{el,m}\chi_{k,m}=E_k\chi_{k,m}
\intertext{And second the Born-Oppenheimer approximation:}
\hat{C}_{pm}=0 \; \forall m,p \; & \rightarrow \; \hat{T}_n\chi_{k,m} +E_{el,m}\chi_{k,m}=E_k\chi_{k,m}
\end{align}
Solution to the nuclear Schr\"odinger equation \ref{eq:entise} leads to the oscillation levels. Hence the nuclei are moving in a field of electrons. The next step to derive molecular dynamics is to approximate the nuclei as classical point particles which is done in the next section.

\subsection{Approximation of nuclei as classical point particles}
A well-known route to extract semiclassical mechanics from quantum mechanics in general starts with rewriting the corresponding wavefunction for nuclei
\begin{equation}
\chi_{k}(\{ \mathbf{R}_I\};t)=A_k(\{ \mathbf{R}_I\};t)\exp\left[ \frac{\mathrm{i}}{\hbar}S_k(\{ \mathbf{R}_I\};t) \right]
\label{ansatz:pointparticle}
\end{equation}
in terms of an amplitude factor $A_k$ and a phase $S_k$ which are both considered to be real and $A_k > 0$ in this polar representation. This ansatz inserted into \ref{eq:nse} choosing electronic state $k$ and remembering that in time dependent case we use $E_k=\mathrm{i}\hbar\frac{\partial}{\partial t}$, results in equations for $A_k$ and $S_k$ respectively. Starting with
\begin{align*}
&\left[ \hat{T}_n+E_{el}\right]A_k\exp(\frac{\mathrm{i}}{\hbar}S_k)=\mathrm{i}\hbar\frac{\partial }{\partial t}A_k\exp(\frac{\mathrm{i}}{\hbar}S_k)\\
\intertext{inserting definition of kinetic operator \ref{def:nuclearkinetic} and apply product rule for time derivative}
&\left[ \sum_I\frac{-\hbar^2}{2M_I}\nabla_I^2+E_{el}\right]A_k\exp(\frac{\mathrm{i}}{\hbar}S_k)=\mathrm{i}\hbar(\frac{\partial A_k}{\partial t}\exp(\frac{\mathrm{i}}{\hbar}S_k)+\frac{\partial S_k}{\partial t}A_k\exp(\frac{\mathrm{i}}{\hbar}S_k)\frac{\mathrm{i}}{\hbar})\\
\intertext{expand the laplace operator}
&\sum_I\frac{-\hbar^2}{2M_I}\left[ A_k\Delta_I\exp(\frac{\mathrm{i}}{\hbar}S_k)+2(\nabla_I\exp(\frac{\mathrm{i}}{\hbar}S_k))\cdot(\nabla_IA_k)+\exp(\frac{\mathrm{i}}{\hbar}S_k)\Delta_IA_k\right]\\
&+E_{el}A_k\exp(\frac{\mathrm{i}}{\hbar}S_k)=\mathrm{i}\hbar\frac{\partial A_k}{\partial t}\exp(\frac{\mathrm{i}}{\hbar}S_k)-\frac{\partial S_k}{\partial t}A_k\exp(\frac{\mathrm{i}}{\hbar}S_k)\\
\intertext{apply gradient rules to the exponential}
&\sum_I\frac{-\hbar^2}{2M_I}\left[ A_k\Delta_IS_k\exp(\frac{\mathrm{i}}{\hbar}S_k)\frac{-1}{\hbar^2}+2\nabla_IS_k\exp(\frac{\mathrm{i}}{\hbar}S_k))\frac{\mathrm{i}}{\hbar}\cdot(\nabla_IA_k)+\exp(\frac{\mathrm{i}}{\hbar}S_k)\Delta_IA_k\right]\\
&+E_{el}A_k\exp(\frac{\mathrm{i}}{\hbar}S_k)=\mathrm{i}\hbar\frac{\partial A_k}{\partial t}\exp(\frac{\mathrm{i}}{\hbar}S_k)-\frac{\partial S_k}{\partial t}A_k\exp(\frac{\mathrm{i}}{\hbar}S_k)\\
\intertext{the exponential can be cancelled out on each side}
&\sum_I\frac{-\hbar^2}{2M_I}\left[ A_k\Delta_IS_k\frac{-1}{\hbar^2}+2\nabla_IS_k\frac{\mathrm{i}}{\hbar}\cdot(\nabla_IA_k)+\Delta_IA_k\right]+E_{el}A_k=\mathrm{i}\hbar\frac{\partial A_k}{\partial t}-\frac{\partial S_k}{\partial t}A_k\\
\intertext{partition the sum to single terms and cancel factors}
&\sum_I\frac{1}{2M_I}A_k\Delta_IS_k+\sum_I\frac{-\mathrm{i}\hbar}{M_I}(\nabla_IS_k)\cdot(\nabla_IA_k)+\sum_I\frac{-\hbar^2}{2M_I}\Delta_IA_k+E_{el}A_k\\
&=\mathrm{i}\hbar\frac{\partial A_k}{\partial t}-\frac{\partial S_k}{\partial t}A_k\\
\intertext{a coefficient comparison yields the final equations}
&\implies\frac{\partial S_k}{\partial t} + \sum_I\frac{1}{2M_I}(\nabla_IS_k)^2+E_{el}=\hbar^2\sum_I\frac{1}{2M_I}\frac{\nabla_I^2A_k}{A_k} \\
&\implies\frac{\partial A_k}{\partial t}+\sum_I\frac{1}{M_I}(\nabla_IA_k)(\nabla_IS_k)+\sum_I\frac{1}{2M_I}A_k(\nabla_I^2S_k)=0
\end{align*}
whereas the equation for the amplitude multiplied by $2A_k$ is also known as a first continuity equation

\begin{align}
&\frac{\partial A_k^2}{\partial t}+\sum_I\frac{1}{M_I}\nabla_I(A_k^2\nabla_IS_k)=0
\intertext{and with the definition of a current density $\mathbf{J}_{k,I}=A_k^2(\nabla_IS_k)/M_I$ and using the probability density function $\rho_k=|\chi_k|^2 \equiv A_k^2$ a second continuity equation}
&\frac{\partial \rho_k}{\partial t}+\sum_I\nabla_I\mathbf{J}_{k,I}=0.
\end{align}


Note that these are independent of $\hbar$. More importantly for the classical limit $\hbar \to 0$ the equation for $S_k$ reduces to:
\begin{equation}
\frac{\partial S_k}{\partial t} + \sum_I\frac{1}{2M_I}(\nabla_IS_k)^2 + E_{el} = 0
\label{eq:classicallimitbo}
\end{equation}
which is equivalent to the Hamilton-Jacobian formulation of classical mechanics
\begin{align*}
&\frac{\partial S_k}{\partial t} + H_k(\{ \mathbf{R}_I \},\{ \nabla_IS_k \}) = 0
\intertext{with the classical Hamilton function}
&H_k(\{ \mathbf{R}_I \},\{\mathbf{P}_I  \}) =T(\{\mathbf{P}_I\})+V_k(\{ \mathbf{R}_I \})
\intertext{for a given conserved energy $d E_{k}^{tot} /dt =0$ and hence}
&\frac{\partial S_k}{\partial t} =-(T+E_{el,k})=-E_k^{tot}=const.
\intertext{defined in terms of generalized coordinates $\{ \mathbf{R}_I \}$ and their conjugate canonical momenta $\{ \mathbf{P}_I \}$. With the help of the connecting transformation}
&\mathbf{P}_I \equiv \nabla_IS_k \left[=M_I\frac{\mathbf{J}_{k,I}}{\rho_k} \right]
\end{align*}
the Newtonian equations of motion corresponding to $\ref{eq:classicallimitbo}$ can be read off
\begin{align}
&\frac{d \mathbf{P}_I}{dt} = -\nabla_IE_{el,k}\text{ or}\nonumber\\
&M_I\ddot{\mathbf{R}}_I = -\nabla_IV_k^{BO}(\{ \mathbf{R}_I\})\label{eq:eombo}
\end{align}
separately for each decoupled electronic state $k$. Thus, the nuclei move according to classical mechanics in an effective potential, $V_k^{BO}$ , which is given by the Born-Oppenheimer potential energy surface $E_k$ obtained by solving simultaneously the time-independent electronic Schrödinger equation for the $k$th state, \ref{eq:tiese}, at the given nuclear configuration $\{\mathbf{R}_I (t)\}$. Note that this result is based on ansatz \ref{ansatz:bo} and \ref{ansatz:pointparticle} but can be obtained directly when using the Lagrangian in \ref{lagrangian:BO}. The significant difference is that the Lagrangian was an educated guess whereas here it is physically motivated and derived.

Vice versa one can insert the point particle ansatz \ref{ansatz:pointparticle} into the TDSCF framework given in equations \ref{eq:tdscfnuclear} and \ref{eq:tdscfelectronic}. In this derivation for simplicity only one term for the $\Phi$ product was used.
\begin{align}
&\mathrm{i}\hbar \frac{\partial \chi}{\partial t} = \hat{T}_n\chi+  \Bra{\Phi} \hat{H}_{el}\Ket{\Phi} \chi\label{eq:tdesescfnuclei}\\
&\mathrm{i}\hbar \frac{\partial\Phi}{\partial t}= \hat{T}_e \Phi + \Bra{\chi} \hat{V}_{n-e} \Ket{\chi} \Phi\label{eq:tdesescfelectric}\\
\intertext{inserting ansatz \ref{ansatz:pointparticle}}
&\mathrm{i}\hbar \frac{\partial A\exp(\frac{\mathrm{i}}{\hbar}S)}{\partial t} = \hat{T}_nA\exp(\frac{\mathrm{i}}{\hbar}S)+  \Bra{\Phi} \hat{H}_{el}\Ket{\Phi} A\exp(\frac{\mathrm{i}}{\hbar}S)\nonumber\\
\intertext{inserting definitions of kinetic operator \ref{def:nuclearkinetic}}
&\mathrm{i}\hbar \frac{\partial A\exp(\frac{\mathrm{i}}{\hbar}S)}{\partial t} = \sum_I \frac{-\hbar^2}{2M_I}\nabla_I^2 A\exp(\frac{\mathrm{i}}{\hbar}S)+  \Bra{\Phi} \hat{H}_{el}\Ket{\Phi} A\exp(\frac{\mathrm{i}}{\hbar}S)\nonumber\\
\intertext{the same reformulation as before can be used}
&\mathrm{i}\hbar\frac{\partial A}{\partial t}-\frac{\partial S}{\partial t}A=\sum_I\frac{1}{2M_I}A\Delta_IS+\sum_I\frac{-\mathrm{i}\hbar}{M_I}(\nabla_IS)\cdot(\nabla_IA)\nonumber\\
&+\sum_I\frac{-\hbar^2}{2M_I}\Delta_IA+\Bra{\Phi} \hat{H}_{el}\Ket{\Phi} A\nonumber\\
&\implies\frac{\partial S}{\partial t} + \sum_I\frac{1}{2M_I}(\nabla_IS)^2+\Bra{\Phi} \hat{H}_{el}\Ket{\Phi}=\hbar^2\sum_I\frac{1}{2M_I}\frac{\nabla_I^2A}{A}\nonumber\\
&\implies\frac{\partial A}{\partial t}+\sum_I\frac{1}{M_I}(\nabla_IA)(\nabla_IS)+\sum_I\frac{1}{2M_I}A(\nabla_I^2S)=0\nonumber
\end{align}
Note the difference that here no electronic states $k$ are used and in the classical limit $\hbar \to 0$
\begin{equation}
\frac{\partial S}{\partial t} + \sum_I\frac{1}{2M_I}(\nabla_IS)^2+\Bra{\Phi} \hat{H}_{el}\Ket{\Phi}=0
\label{eq:classicallimitscf}
\end{equation}
Correspondingly, the Newtonian equations of motion equivalent to \ref{eq:classicallimitscf} are given by
\begin{align}
&\frac{d \mathbf{P}_I}{dt} = -\nabla_I\Bra{\Phi} \hat{H}_{el}\Ket{\Phi}\text{ or}\nonumber\\
&M_I\ddot{\mathbf{R}}_I = -\nabla_IV_e^{E}(\{ \mathbf{R}_I\})\label{eq:eomscf}
\end{align}
Here, the nuclei move according to classical mechanics in an effective potential $V_e^{E}$, often called the Ehrenfest potential, which is given by the quantum dynamics of electrons obtained by solving simultaneously the time-dependent electronic Schr\"odinger equation \ref{eq:tdesescfelectric}. Note that this interaction potential due to explicit time evolution of the quantum electrons stems from averaging the electronic Hamiltonian with respect to electronic degrees of freedom $V_e^{E} = \Bra{\Psi}\hat{H}_{el}\Ket{\Psi}$. However equation \ref{eq:tdesescfelectric} still includes the full quantum-mechanical nuclear wavefunction $\chi(\{\mathbf{R}_I\};t)$ instead of just the classical-mechanical nuclear positions $\{\mathbf{R}_I\}$. This can be reduced in the classical limit $\hbar \to 0$, because the nuclear density function $|\chi(\{\mathbf{R}_I\};t)|^2$ in equation \ref{eq:tdesescfelectric} can be rewritten as a product of delta functions $\prod_I\delta(\mathbf{R}_I -\mathbf{R}_I(t))$ centered at the instantaneous positions $\{\mathbf{R}_I\}$ of the classical nuclei. Hence,
\begin{align*}
\int_{}^{} \chi^*(\{\mathbf{R}_I\};t) \mathbf{R}_I \chi(\{\mathbf{R}_I\};t)d\mathbf{R} \xrightarrow[]{\hbar \to 0} \mathbf{R}_I(t)
\intertext{and therefore}
\mathrm{i}\hbar \frac{\partial\Phi}{\partial t}= -\sum_i\frac{\hbar^2}{2m_e}\nabla_i \Phi + \hat{V}_{n-e}(\{\mathbf{r}_i\},\{\mathbf{R}_I(t)\})\Phi\\
=\hat{H}_{el}(\{\mathbf{r}_i\},\{\mathbf{R}_I(t)\})\Phi(\{\mathbf{r}_i\},\{\mathbf{R}_I\};t)
\end{align*}
This means that feedback between the classical and quantum degrees of freedom is incorporated in both directions, although in a mean-field sense only. This approach is often called "Ehrenfest molecular dynamics" in honor of Paul Ehrenfest who was the first to address the essential question of how Newtonian classical dynamics of point particles can be derived from Schr\"odiner's time-dependent wave equation \cite{ehrenfest1927bemerkung}.

\subsection{Recovering BOMD from EhrenfestMD}
This can be done by expanding the electronic wave function $\Phi$ again into a basis of electronic states:
\begin{align*}
\Phi(\{ \mathbf{r}_i\},\{ \mathbf{R}_I\};t ) = \sum_{l=0}^{\infty} c_l(t)\Phi_l(\{ \mathbf{r}_i\};\{ \mathbf{R}_I\})
\end{align*}
with complex time-dependent coefficients $\{ c_l(t)\}$

\subsection{Recapitulation}
\subsection{Further Possibilities}
extended lagrangian bomd, fock dynamics,
\subsection{Conclusion}

\newpage
\begin{landscape}
\begin{figure}
\includegraphics[scale=1.0]{tikz.pdf}
\end{figure}
%\includestandalone{tikz}
\end{landscape}
\newpage

cites...
\nocite{car1985unified} %car parinello MD
\nocite{makri1987time} %tdscf
\nocite{marx2009ab} %wichtigstes BUCH
\nocite{niklasson2009extended} %extended lagrangian bomd
\nocite{payne1992iterative} %iterative minimization techniques for ab inito
\nocite{pulay2004fock} %fock matrix dynamics
\nocite{tangney2002well} %car parinello vs bo surface
\nocite{tangney2006theory} %theory car parinello fictous mass

Iterative solution takes computational a lot of effort -> use information of the previous timestep to start optimization

-> introduce time fictitious dependence on previous timestep for good inital guess

unphysical since electron have no memory


Choice of basis? Atomic basis(physical), Gaussian type basis, Plane wave basis with pseudo potentials(see CP)

It is of importance to note that depending on the choice of view on orbitals different thinking applies. When the person looking for a method to simulate a system also wants that the properties the algorithm uses have physical/chemical meaning then one should use Born-Oppenheimer molecular dynamics with a basis representation with atomic orbitals e.g. Gaussian type orbitals. Is the person also interested in methods with time reversal also more attention have to be paid to unphysical variables since electronic states don't have a "memory". The basic approach is that, in every time step look for the minimum on the electronic surface potential for the current nuclei configuration, and use the gradient to calculate the force used to move the nuclei to the next position. Another approach is to allow oscillation in the electronic degrees of freedom like in the Car-Parinello molecular dynamics 






note:(Fock matrix)
Car Parinello used local DFT: atomic cores -> pseudo potentials
							  valence electrons -> plane wave basis set
which allowed to use FFT for evaluating Coulomb operator(most expensive part). Replacing the exact forces with Hellman force.  Propagation of wavefunction rather than optimizing in each step (BOMD). CPMD to sensible on choosing correct fictitious mass which leads to artifacts. Proposal is to extrapolate Fock matrix in time instead.




Kohn-Sham equations and fictious potentail
Kohn-Sham orbitals are not unique
problem of Energy drift
artifacts because of dynamic propagation of electronic orbitals
how to handle boundary condition?


\newpage
\bibliographystyle{plain}
\bibliography{references}


\end{document}
