\documentclass[a4paper, oneside]{discothesis}

% use utf8 instead of latin1 when using LaTeX in windows
\usepackage[latin1]{inputenc}


%%%%%%%%%%%%%%%%%%%%%%%%%%%%%%%%%%%%%%%%%%%%%%%%%%%%%%%%%%%%%%%%%%%%%%%%%%%%%%%%%%%%%%%%%%%%%%%%%
% DOCUMENT METADATA

\thesistype{Master Thesis}
\title{Adabtive Hierarchical Deep Reinforcement Learning}

\author{Florian Frei}
\email{flofrei@student.ethz.ch}
\institute{Distributed Computing Group \\[2pt]
Computer Engineering and Networks Laboratory \\[2pt]
ETH Z�rich}

% You can put in your own logo here "\includegraphics{...}" or just comment the command
\logo{}

\supervisors{Gino Brunner, Oliver Richter\\[2pt] Prof.\ Dr.\ Roger Wattenhofer}

% You can comment the following two commands if you don't need them
% \keywords{Keywords go here.}
% \categories{ACM categories go here.}

\date{\today}

%%%%%%%%%%%%%%%%%%%%%%%%%%%%%%%%%%%%%%%%%%%%%%%%%%%%%%%%%%%%%%%%%%%%%%%%%%%%%%%%%%%%%%%%%%%%%%%%%

\begin{document}

\frontmatter % do not remove this line
\maketitle

\cleardoublepage

\begin{acknowledgements}
	I thank Lorem ipsum dolor sit amet, consetetur sadipscing elitr, sed diam nonumy eirmod tempor invidunt ut labore et dolore magna aliquyam erat, sed diam voluptua. At vero eos et accusam et justo duo dolores et ea rebum. Stet clita kasd gubergren, no sea takimata sanctus est Lorem ipsum dolor sit amet. Lorem ipsum dolor sit amet, consetetur sadipscing elitr, sed diam nonumy eirmod tempor invidunt ut labore et dolore magna aliquyam erat, sed diam voluptua. At vero eos et accusam et justo duo dolores et ea rebum. Stet clita kasd gubergren, no sea takimata sanctus est Lorem ipsum dolor sit amet.
\end{acknowledgements}


\begin{abstract}
	Lorem ipsum dolor sit amet, consetetur sadipscing elitr, sed diam nonumy eirmod tempor invidunt ut labore et dolore magna aliquyam erat, sed diam voluptua. At vero eos et accusam et justo duo dolores et ea rebum. Stet clita kasd gubergren, no sea takimata sanctus est Lorem ipsum dolor sit amet. Lorem ipsum dolor sit amet, consetetur sadipscing elitr, sed diam nonumy eirmod tempor invidunt ut labore et dolore magna aliquyam erat, sed diam voluptua. At vero eos et accusam et justo duo dolores et ea rebum. Stet clita kasd gubergren, no sea takimata sanctus est Lorem ipsum dolor sit amet.
\end{abstract}

\tableofcontents

\mainmatter % do not remove this line

% Start writing here
\chapter{Overview}
$\bullet$ basic actor critic with td error on an 1D environment multithreaded \\
$\bullet$ learning tensorflow with python \\
$\bullet$ options paper \\
$\bullet$ learning about gym for definition of own environment \\
$\bullet$ implement option critic on multithreaded own environment \\

fast and easy environment for testing the abstraction levels and depth



\chapter{Motivation}
The human capability to learn things is well-known but not well understood. Our brain is the most complex organ in our body and its size is the most significant difference to other species. It allows us to be self-aware and defines our individuality. Hence it is of significant importance to understand how it works for creating artificial intelligence. One aspect of it is the brains capability of high level abstraction. This could be temporal or objective in nature. Meaning that we learn automatically to tackle difficult tasks by dividing them into manageable subtasks. Additionally the function of dopamine allows us to handle delayed gratification. Meaning that for example no immediate reward is needed for pursuing a goal in life. The journey is a sufficient inner motivator regardless if we can reach the set goal. The high complexity of games in the world today such as league of legends, dota or starcraft require a more sophisticated abstraction of the strategy in order to win. Thus it is of importance to understand how abstraction can be achieved by using deep neural networks. 

\chapter{Introduction}
%Reinforcment Learning
Reinforcement Learning(RL) is besides supervised and unsupervised learning one of the big three themes in machine learning. Without going into details this research field is mostly based on the theory of Markov and semi-Markov decision processes. There are other types of approaches worth mentioning but we don't want to go into to much detail. For the interested reader on the history and formation of reinforcement learning we refer to the well-known survey of Kaelbling et al. (1996) \cite{kaelbling1996reinforcement} were everything can be found. In this work we concentrate on one aspect of reinforcement learning namely hierarchical reinforcement learning. We think a good start where most frameworks are well summarized is in Barto et al.(2003) \cite{barto2003recent} and it also gives a brief introduction to Markov decision processes.
Another noteworthy introduction we think is the reinforcement course by David Silver free available online on youtube. \\
%Hierarchical reinforcement learning
The idea of using a hierarchy in reinforcement learning to increase efficiency is well established. The first approaches was to define macro actions or operators which can be invoked instead of just basic actions. The first three real frameworks which are still used today were proposed in the nineties. To summarize these three major frameworks were the options framework of Sutton et al. (1999) \cite{sutton1999between}, the HAM(hierarchical abstract machines) framework of Parr et al. (1998) \cite{parr1998reinforcement}, and the MaxQ framework of Dietterich (2000) \cite{dietterich2000hierarchical}.
This work is based on the options framework and hence we concentrate only on this type of framework but this does not mean the others are not worth consideration. \\
Another big aspect in the development of machine learning is due to the rapid improvement of the hardware. More memory and CPU capacity allows faster exploration and tackling more complex tasks. This allowed the realization of neural networks. Neural networks also have a rich history of development and we refer for further reading to Funahashi (1989) \cite{funahashi1989approximate} where one can get an impression on the history. Neural networks form the basis of most deep reinforcement algorithms today and also used for this work. A good starting point into this topic is the work of Mnih et al. (2015) \cite{mnih2015human} which shows that a high level of control can be achieved. 
 

At the beginning an experience buffer was used to store all transitions in a given environment.
the states, actions, and rewards were saved to a memory location.  



experience replay vs asynchronous training
Q learning, policy gradient, monte carlo roll out
on vs off policy algorithms
eligible traces, temporal difference errors, efficient data sampling  


\cite{mnih2016asynchronous} 



Options can be looked at as an encapsulated policy
The problem with micro actions generally is to tell when to apply a micro action and when to stop
An options is composed of an initation set, a policy and a termination function

asynchronous advange actor critic (A3C) 

asynchronous advantage option critic architecture (A2OC)

temporally abstracted actions


tensorflow library \cite{abadi2016tensorflow}
activity in HAM \cite{bai2017efficient}

\chapter{Preliminaries}

\chapter{Baseline}

\chapter{Actor Critic}

\chapter{Option Critic}
\TODO{Not finished}

\chapter{Experiments}




\section{Thoughts}

The option critic has a significant overhead in comparison to the actor critic implementation.
The option critic architecture consists of 3 network types, namely an option network, a Q value model and a termination model.
The actor critic model as the name implies consists only of an actor and a critic model hence two networks.
The Q value model describe how a state is for the given options.
The options describe action probabilities for a given state and can be understood as a strategy.
The termination model characterized how likely a given option terminates given the state as input. 

\begin{theorem}[First Theorem] \label{thm:first theorem}
	This is our first theorem.
\end{theorem}

\begin{proof}
	And this is the proof of the first theorem with a complicated formula and a reference to Theorem \ref{thm:first theorem}. Lorem ipsum dolor sit amet, consetetur sadipscing elitr, sed diam nonumy eirmod tempor invidunt ut labore et dolore magna aliquyam erat, sed diam voluptua. Lorem ipsum dolor sit amet, consetetur sadipscing elitr, sed diam nonumy eirmod tempor invidunt ut labore et dolore magna aliquyam erat, sed diam voluptua.
	\begin{equation}
		{\frac {\mathrm d}{\mathrm dx}}\arctan(\sin({x}^{2}))=-2 \cdot {\frac {\cos({x}^{2})x}{-2+\left (\cos({x}^{2})\right )^{2}}}
	\end{equation}	
\end{proof}

And here we cite an external document~\cite{TestReference}.

% This displays the bibliography for all cited external documents. All references have to be defined in the file references.bib and can then be cited from within this document.
\bibliographystyle{splncs}
\bibliography{references}

% This creates an appendix chapter, comment if not needed.
\appendix
\chapter{Appendix Chapter}

\end{document}