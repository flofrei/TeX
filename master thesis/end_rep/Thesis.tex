\documentclass[a4paper, oneside]{discothesis}

% use utf8 instead of latin1 when using LaTeX in windows
\usepackage[latin1]{inputenc}
\usepackage{mathtools}

\DeclareMathOperator*{\argmax}{\operatorname{argmax}}

%%%%%%%%%%%%%%%%%%%%%%%%%%%%%%%%%%%%%%%%%%%%%%%%%%%%%%%%%%%%%%%%%%%%%%%%%%%%%%%%%%%%%%%%%%%%%%%%%
% DOCUMENT METADATA

\thesistype{Master Thesis}
\title{Adabtive Hierarchical Deep Reinforcement Learning}

\author{Florian Frei}
\email{flofrei@student.ethz.ch}
\institute{Distributed Computing Group \\[2pt]
Computer Engineering and Networks Laboratory \\[2pt]
ETH Z�rich}

% You can put in your own logo here "\includegraphics{...}" or just comment the command
\logo{}

\supervisors{Gino Brunner, Oliver Richter\\[2pt] Prof.\ Dr.\ Roger Wattenhofer}

% You can comment the following two commands if you don't need them
% \keywords{Keywords go here.}
% \categories{ACM categories go here.}

\date{\today}

%%%%%%%%%%%%%%%%%%%%%%%%%%%%%%%%%%%%%%%%%%%%%%%%%%%%%%%%%%%%%%%%%%%%%%%%%%%%%%%%%%%%%%%%%%%%%%%%%

\begin{document}

\frontmatter % do not remove this line
\maketitle

\cleardoublepage

\begin{acknowledgements}
	I thank Lorem ipsum dolor sit amet, consetetur sadipscing elitr, sed diam nonumy eirmod tempor invidunt ut labore et dolore magna aliquyam erat, sed diam voluptua. At vero eos et accusam et justo duo dolores et ea rebum. Stet clita kasd gubergren, no sea takimata sanctus est Lorem ipsum dolor sit amet. Lorem ipsum dolor sit amet, consetetur sadipscing elitr, sed diam nonumy eirmod tempor invidunt ut labore et dolore magna aliquyam erat, sed diam voluptua. At vero eos et accusam et justo duo dolores et ea rebum. Stet clita kasd gubergren, no sea takimata sanctus est Lorem ipsum dolor sit amet.
\end{acknowledgements}


\begin{abstract}
	Lorem ipsum dolor sit amet, consetetur sadipscing elitr, sed diam nonumy eirmod tempor invidunt ut labore et dolore magna aliquyam erat, sed diam voluptua. At vero eos et accusam et justo duo dolores et ea rebum. Stet clita kasd gubergren, no sea takimata sanctus est Lorem ipsum dolor sit amet. Lorem ipsum dolor sit amet, consetetur sadipscing elitr, sed diam nonumy eirmod tempor invidunt ut labore et dolore magna aliquyam erat, sed diam voluptua. At vero eos et accusam et justo duo dolores et ea rebum. Stet clita kasd gubergren, no sea takimata sanctus est Lorem ipsum dolor sit amet.
\end{abstract}

\tableofcontents

\mainmatter % do not remove this line

% Start writing here
\chapter{Motivation}
The human capability to learn things is well-known but not well understood. Our brain is the most complex organ in our body and its size is the most significant difference to other species. It allows us to be self-aware and defines our individuality. Hence it is of significant importance to understand how it works for creating artificial intelligence. One aspect of it is the brains capability of high level abstraction. This could be temporal or objective in nature. Meaning that we learn automatically to tackle difficult tasks by dividing them into manageable subtasks. Additionally the function of dopamine allows us to handle delayed gratification. Meaning that for example no immediate reward is needed for pursuing a goal in life. The journey is a sufficient inner motivator regardless if we can reach the set goal. The high complexity of games in the world today such as league of legends, dota or starcraft require a more sophisticated abstraction of the strategy in order to win. Thus it is of importance to understand how abstraction can be achieved by using deep neural networks. 

\chapter{Introduction}
%Reinforcment Learning
Reinforcement Learning(RL) is besides supervised and unsupervised learning one of the big three themes in machine learning. Without going into details this research field is mostly based on the theory of Markov Decision Processes. There are other types of approaches worth mentioning but we don't want to go into to much detail. For the interested reader or for a complete beginner in this kind of field we refer to we refer Sutton et al. (1998) \cite{sutton1998introduction} and Kaelbling et al. (1996) \cite{kaelbling1996reinforcement} where most on theory and history of this research field with all references can be found.\\
In this work we concentrate on one subfield of reinforcement learning namely hierarchical reinforcement learning. We think a good start where most frameworks are well summarized is in Barto et al.(2003) \cite{barto2003recent} and it also gives a brief introduction to the theory of Markov Decision Processes abbreviated as MDP. For the interested reader which is not particularly fond of reading papers we refer to video course by David Silver free available online on youtube.\\
%Hierarchical reinforcement learning
The idea of using a hierarchy in reinforcement learning to increase efficiency is well established. The first approaches was to define macro actions or operators which can be invoked instead of just basic actions. The first three real frameworks which are still used today were proposed in the nineties. To summarize these three major frameworks were the options framework of Sutton et al. (1999) \cite{sutton1999between}, the HAM(hierarchical abstract machines) framework of Parr et al. (1998) \cite{parr1998reinforcement}, and the MaxQ framework of Dietterich (2000) \cite{dietterich2000hierarchical}. This thesis is based on the options framework and hence we concentrate only on this type of framework but this does not mean the others are not worth consideration. \\
Another big aspect in the development of machine learning is due to the rapid improvement of the hardware. More memory and CPU capacity allows faster exploration and tackling more complex tasks. This allowed the realization of neural networks and hence the field of deep reinforcement learning. For a reader which never heard of neural networks we refer to Funahashi (1989) \cite{funahashi1989approximate} where the theory and research references can be found. Neural networks form the basis of most deep reinforcement algorithms today and are used extensively. 

%Bad start reference to randomly sampled replay buffer in RL
%A good starting point into this topic is the work of Mnih et al. (2015) \cite{mnih2015human} which shows that a high level of control can be achieved. 

The state of the art algorithm A3C(asynchronous advantage actor critic) can be found in Mnih et al.(2016) \cite{mnih2016asynchronous} which is often used as a baseline to compare the performance of new developed techniques and algorithms.

% The toolkit we use to compare algorithms fairly on different environments is given by the OpenAI gym from Brockman et al. \cite{brockman2016openai} and will be used as a base in this work to implement our own environments. \\
%To implement neural networks we refer to two libraries namely the overworked theano framework \cite{bergstra2011theano} from Bastien et al. (2012) \cite{bastien2012theano} and tensorflow from Abadi et al. (2016) \cite{abadi2016tensorflow}. In this work we will exclusively use tensorflow in python.
In this work we concentrate on the proposed hybrid algorithm between the options framework and asynchronous advantage actor critic named option critic from \cite{harb2017waiting} where a deliberation cost was additionally introduced.

\chapter{Preliminaries}
%For the complete theory on MDP's we refer to the literature.

Most theory of reinforcement learning research is based on Markov Decision Processes(MDP) since it provides the simplest framework which allows us to study basic algorithms and their properties. A finite discounted MDP $\mathcal{M}$ is a tuple $\mathcal{M} \doteq (\mathcal{S},\mathcal{A},\gamma,r,P)$ encapsulating a state space $\mathcal{S}$, an action space $\mathcal{A}$, a discount factor $\gamma$ where $\gamma \in [0,1)$, a reward function $r(s,a): \mathcal{S} \times \mathcal{A} \to \mathbb{R}$ which maps a state-action pair to a real number and state transition probability function $P(s,a):\mathcal{S} \times \mathcal{A} \to \mathbb{S} $ which maps old states with actions to new states.\\
A policy $\pi(s) : \mathcal{S} \to \mathcal{A} $ is a description of behavior. Meaning in a deterministic setting the action taken is always the same given the same state. In a stochastic setting the policy $\pi(a | s) : \mathcal{S} \times \mathcal{A} \to [0,1]$ is a distribution given a state. This means an action is sampled from this distribution where each action has a probability of getting selected. \\
Since a reward can be assigned to each state a value function can be calculated from start state $s$ with a given stochastic policy as follows:
\begin{equation*}
V^{\pi}(s) = \mathbb{E} \left[ \sum_{t=0}^{\infty} \gamma^{t} r_t(s_t,a=\pi(s_t)) \right]
\end{equation*}
This value can be used to compare different policies given start state $s$. With the same concept in mind a value can be calculated when the first action is additionally separated from the rest of the policy. This yields a state-action function $Q_{\pi}(s,a):\mathcal{S}\times\mathcal{A} \to \mathbb{R}$ which can be used to compare different initial actions $a$ from a given start state $s$.\\
Now instead looking at the whole trajectory of a policy the Bellman equation \eqref{eq:bellman} from \cite{bellman1956dynamic} (1956) allows one to view it in a recursive manner using the value function as follows:
\begin{equation}
\label{eq:bellman}
V^{\pi} (s)= r(s,a=\pi(s)) + \gamma \sum_{s' \in \mathcal{S}}^{}{ P(s'|s,a=\pi(s)) V^{\pi}(s') }
\end{equation}
In the stochastic setting when the policy follows a distribution and every event is possible the equation can be rewritten as follows:
\begin{equation}
\label{eq:bellman2}
V^{\pi} (s)= \sum_{a}^{}{\pi(a | s)} \left( r(s,a) + \gamma \sum_{s'}^{}{ P(s'|s,a) V^{\pi}(s') } \right) 
\end{equation}
This equation was already usable in dynamic programming. Two major algorithms using Bellman equation were value iteration and policy iteration from Howard (1964) \cite{howard1964dynamic} which could solve successfully MDP's. In value iteration the algorithm acts under an implicit policy such as $\epsilon$-greedy for example and the value function is learned through the state-action function. 
\begin{gather*}
Q_{t+1}(s,t) = r(s,a) + \gamma \sum_{s'}^{}{P(s'|s,a)V_{t}(s)} \\
V_{t+1}(s) = \max_{a} Q_{t+1}(s,a)
\end{gather*}
Whereas in policy iteration a random policy $\pi$ is initialized and then learned stepwise using the value function but not the state-action function.
\begin{gather*}
V^{\pi_t}_{t+1}(s) = r(s,\pi_t(s))+\gamma \sum_{s' \in \mathcal{S}}^{}{P(s'|s,\pi_t(s))V^{\pi_t}_{t+1}(s')}\\
\pi_{t+1}(s) = \argmax_{a} \left( r(s,a) + \gamma \sum_{s'}^{}{P(s'|s,a)V^{\pi_t}_{t+1}(s')} \right)
\end{gather*}
As one can imagine in reinforcement learning the Bellman equation allows different lengths of recursion. An algorithm using the whole trajectory in an environment are categorized under Monte Carlo methods. The alternative is Temporal Difference (TD) \cite{sutton1988learning} methods which use the Bellman equation and instead of calculating the expectation they estimate the difference from sampling. Hence the temporal difference error resulting is written as:
\begin{equation*}
\operatorname{TD}(0) = R(s,\pi(s))+\gamma V_{\pi}(s') -\gamma V_{\pi}(s) 
\end{equation*}
Notice that $s'$ was sampled from the transition probabilities and can be looked at as a zero order scheme. One can repeat the process and use the same equation to approximate the next value $V_{\pi}(s')$ and so forth. Thus arbitrary recursion schemes can be chosen for this approximation which is a design decision depending on the environment or problem respectively.\\
In this recursion we only talked about improving the value function but we are also interested in improving the policy $\pi$ along the way. This can be done in two manners namely on-policy or off-policy. On-policy means that only one policy is used and directly improved with the environment. Off-policy means that there are two policies. One used for exploration with generates data whereas the other policy uses this data to improve itself. Since the policies have different acting distributions sampling methods like importance sampling has to be used to eliminate bias from the action and state samples. This means that in the off-policy case an experience buffer is filled with transitions and then sampled from it to improve the policy.\\
An example for an on-policy algorithm is SARSA \cite{rummery1994line} which stands abbreviated for state action reward state action meaning it uses one transition and the next action to approximate the TD-error. The policy used is $\epsilon$-greedy based on the state-action function $Q$. Meaning with probability $\epsilon$ a random action is taken otherwise the action with the maximum $Q$-value is used. The counterpart to SARSA using off-policy updates is Q-Learning \cite{watkins1992q}.\\
As we have seen either $\epsilon$-greedy policy was used based on action-value function $Q(s,a)$ to act or another entire policy based on experience. The functions $V(s)$ and $Q(s,a)$ helped to decide which actions are better in the environment in given state and hence improve the policy. But the policy also can be learned directly and are named policy gradient methods. The first method introduces was REINFORCE from Williams (1992) \cite{williams1992simple} where a whole trajectory like in the Monte Carlo method was utilized to improve the policy. The objective is still to maximize the expected cumulative reward.
\begin{equation*}
\rho(s) = \mathbb{E} \left[ \sum_{t=0}^{\infty}{\gamma^{t-1} R_t \Bigg| s} \right]
\end{equation*}
When we define $\tau$ as a complete trajectory and $R(\tau)$ as the reward corresponding to this trajectory the equation can be reformulated as:
\begin{equation*}
\rho(s) = \sum_{\tau}^{}{P_{\theta}(\tau) R(\tau)}
\end{equation*}
where $\theta$ represent the parameters of our policy $\pi$. Instead of using the whole trajectory one can again use an value estimation. This can be learned with TD and reformulated as maximizing:
\begin{equation*}
\rho(s) = \mathbb{E} \left[ \sum_{t=0}^{\infty}{\gamma^{t-1} R_t \Bigg| s} \right] = \sum_{a}^{}{\pi(a|s)Q^{\pi}(s,a)=V^{\pi}(s)}
\end{equation*}
We will not go into detail for the derivation and refer to Sutton et al. (2000) \cite{sutton2000policy} and just present the resulting gradient.
\begin{equation*}
\frac{\partial V^{\pi}(s)}{\partial \theta} = \sum_{x}^{}{d^{\pi}(x)} \sum_{a}^{}{\frac{\partial \pi(a|x) }{\partial \theta} Q^{\pi}(x,a)} 
\end{equation*}
where $d^{\pi}(s)$ is defined as the discounted weighting of states encountered starting at some state $s_0$. In practice most algorithms use $d^{\pi}(x)$ as a stationary distribution which results in an expectation over states. Additionally a baseline is added to reduce variance and together with the log transformation trick the sum can be reduced as follows:
\begin{gather*}
\frac{\partial V^{\pi}(s)}{\partial \theta} = \sum_{s}^{} d^{\pi}(s) { \sum_{a}^{} { \frac{\partial \pi(a|s) }{\partial \theta}  \left[ Q^{\pi}(s,a) - V(s) \right] } } \\
=\sum_{s}^{}{d^{\pi}(s)} \mathbb{E} \left[  \frac{\partial \log{ \pi(a|s) } }{\partial \theta} \left( Q^{\pi}(s,a) - V(s) \right) \right]
\end{gather*}
Note that $V(s)$ in the expectation does not contribute anything so $A(s,a) = Q^{\pi}(s,a) - V(s)$ is often used and known as advantage function. The actor-critic algorithm is composed of an actor which learns a policy and a critic which learns a value function. The policy is learned similar to REINFORCE but learned with the value function instead of a sample trajectory. The value function off the current policy is learned via TD approximation. The resulting equations are listed below.  
\begin{gather*}
\frac{\partial \rho}{\partial \vartheta} = \sum_{s} d^{\pi}(s) \sum_{a} \frac{\partial \pi(s,a)}{\partial \vartheta } Q^{\pi}(s,a)= \sum_{s} d^{\pi}(s) \mathbb{E}_a \left[  \frac{\partial \log(\pi(a|s) ) }{\partial \vartheta } Q^{\pi}(s,a) \right] \\
\frac{\partial Q(s,a)}{\partial \theta} = \frac{\partial \left( R(s,a) + \gamma \mathbb{E}_{s'\sim P(s'|s,a)}[V(s')]-Q(s,a) \right)^2 }{\partial \theta} = - \sigma \ast \frac{\partial Q(s,a)}{\partial \theta}
\end{gather*}

\TODO{graphic for actor critic}

The next step is to incorporate temporal abstraction into an algorithm. This is no longer possible with simple MDP's and thus the concept was generalized to Semi Markov Decision Processes(SMPD) and we refer to \cite{puterman2014markov} for further reading.

\TODO{SMPD graphic}

This is needed since options in the options framework from Sutton et al. (1999) \cite{sutton1999between} depend on past decisions. Options are in essence encapsulated policies and thus also a super-policy is required who decides which options gets control in a given state. The last piece needed is a function which describes when a sub-policy should terminate itself and give control back to the super-policy. Mathematically described an option consists of three components. A stochastic sub-policy $\pi_{\omega}:\mathcal{S}\times \mathcal{A} \to [0,1]$ which represents the behavior of an option $\omega$, a termination function $\beta: \mathcal{S} \to [0,1]$ which describes the probability of terminating option $\omega$ given state $s$, and an initiation set $\mathcal{I}_{\omega} \subset \mathcal{S}$ containing the all states from which an option $\omega$ can start from. Last but not least the super-policy $\pi_{\Omega}: \mathcal{S} \times \Omega \to [0,1]$ where $\Omega$ is the set of options. The super policy can again be an $\epsilon$-greedy policy.%but now we act greedily based on the estimated $Q$-value belonging to the options. 

\TODO{graphic option critic}

The option-critic architecture is similar to the actor-critic algorithm hence the name. For a complete mathematical derivation we refer to Bacon et al. (2017) \cite{bacon2017option} where everything can be found and we only highlight the resulting equations. In this architecture the super-policy acts $\epsilon$-greedy based on $Q$-values learned by $Q$-learning. The super-policy selects an option which is as already mentioned just a stochastic sub-policy and returns basic actions until termination. As one can imagine this model allows in a sense temporally extended sub-strategies. When the termination function is to high for all states the algorithm collapses to basic actions but chosen indirectly over options. The initiation set is generally given by the set of all states meaning any option can start from any state. The function for the $Q$-values used by the super-policy is given by the following equation:
\begin{equation*}
Q_{\Omega}(s,\omega) = \sum_{a} \pi_{\omega,\theta}(a|s) Q_{U}(s,\omega,a)
\end{equation*}
It is simply the expectation over all actions an option $\omega$ can take and a corresponding state-option-action value given by the sub-policy. The state-option-action value function is defined as follows:
\begin{gather*}
Q_{U}(s,\omega,a) r(s,a) + \gamma \sum_{s'} P(s'|s,a) U(\omega,s') \\
U(\omega,s') = (1-\beta_{\omega,\vartheta}(s'))Q_{\Omega}(s',\omega) + \beta_{\omega,\vartheta}(s') V_{\Omega}(s')
\end{gather*}
where $U(\omega,s')$ is an utility term calculated for the two cases. When the option $\omega$ does not terminate we use the $Q$-value from before for the new state $s'$ otherwise use an estimation $V_{\Omega}(s')$ over all options. Note that this recursion has the same structure as the Bellman equation and hence TD estimation is also applicable.  



\chapter{Experiments}
many experiments
much data

changed experiment environment domain
behaviour of options of different kind of structures on this domain
overfitting prolem? no termination
learning of options vs superpolicy
delibration cost not needed?

different hyperparameters and their influence

filter size
world size
learning rate
annealing vs 0.1 eps

bias part

learning missing knowledge

missing 1 direction
missing 2 direction parallel vs orthogonal
missing 0 direction
superpolicy speed

2D key environment
4 2 layered options
3 2 layered options with 1 expert
expert end termination problem

jump vs bad ending for expert
reward 1 vs 0 on key
local optima prison
-delib cost annealing


\section{Small vs longer rollout}
Multiple threads create local batches of data. One way is to create whole trajectories with the environment and then apply the update gradients on the network. These types of methods are called Monte Carlo methods. This is only possible when the trajectory is small enough since the longer the trajectory the smaller the probability of it occurring and hence more difficult using this data to train. This was ensured by using the termination of the environment. When the agent walks into the boundary of the domain the environment terminated the agent. This accomplished small local batches of data where the agent could learn efficiently. In case where the agent was not terminated learning was close to impossible with the generated data. Another way is make small local steps in the environment and then update the network with this local gradient. This methods are called Temporal Difference methods.


\nocite{merheb2017learning}

\section{Environment}
The toolkit we use to implement our own test environments is the OpenAI gym from Brockman et al. \cite{brockman2016openai} in python.

1D 2D 2Dkey

\section{Networks}
hyoerparameters? 


\section{Asynchronous Advantage Option Critic (A2OC)}


Bacon et al. 2017 \cite{bacon2017option} has shown that learning options in an end-end fashion is possible but frequent termination is an issue. The introduction of a delibration cost in \cite{harb2017waiting} helped but again is a sensitive hyperparameter. 


In case that the environment is changed to something simpler than ALE



\chapter{Thoughts}
$\bullet$ basic actor critic, td error, 1D environment, multithreaded \\
$\bullet$ learning tensorflow with python \\
$\bullet$ options paper \\
$\bullet$ learning about gym for definition of own environment \\
$\bullet$ implement option critic on multithreaded own environment \\

fast and easy environment for testing the abstraction levels and depth


The option critic has a significant overhead in comparison to the actor critic implementation.
The option critic architecture consists of 3 network types, namely an option network, a Q value model and a termination model.
The actor critic model as the name implies consists only of an actor and a critic model hence two networks.
The Q value model describe how a state is for the given options.
The options describe action probabilities for a given state and can be understood as a strategy.
The termination model characterized how likely a given option terminates given the state as input. 


% This displays the bibliography for all cited external documents. All references have to be defined in the file references.bib and can then be cited from within this document.
\bibliographystyle{splncs}
\bibliography{references}

% This creates an appendix chapter, comment if not needed.
\appendix
\chapter{Appendix Chapter}
additional experiments

\end{document}