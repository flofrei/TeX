\begin{abstract}
It is of general interest in physics and chemistry to find viable algorithms to solve the time dependent Schr\"odinger equation. This is in most cases not directly possible so different approximations have to be made. Such a one is the semiclassical approach with \textit{Hagedorn} wavepackets. To test this approach for reliability a python implementation was created. With this implementation it is possible to test infinite many starting configuration which results in a lot of data. To handle this huge amount of data a binary format is required because only this format can be efficiently compressed. Hence it is a good idea to use a well-known and common used binary data format such as \textit{HDF5} which also has a reasonable python interface. Since not all problems are solvable in a short period of time, and also due to the fact that python is an interpreted language, the code was ported into C++ to reduce execution time. Therefore also the io-operations have to be ported to C++. Luckily \textit{HDF5} also has a C++ interface. This interface however is not as easy comprehensible as the python interface and thus this thesis explains its core functionality for this application. Furthermore since the same hierarchy is implemented as in python the generated data is directly comparable. This data test was done in C++ with the well-known \textit{GoogleTest} framework.
\end{abstract}
