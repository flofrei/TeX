\chapter{Appendix}

\section{Code of 2D harmonic oscillator}
\begin{lstlisting}
//Standard libraries
#include <iostream>
#include <fstream>
//This framework
#include "waveblocks/types.hpp"
#include "waveblocks/wavepackets/shapes/tiny_multi_index.hpp"
#include "waveblocks/potentials/potentials.hpp"
#include "waveblocks/potentials/bases.hpp"
#include "waveblocks/wavepackets/hawp_paramset.hpp"
#include "waveblocks/wavepackets/hawp_commons.hpp"
#include "waveblocks/wavepackets/shapes/shape_enumerator.hpp"
#include "waveblocks/wavepackets/shapes/shape_hypercubic.hpp"
#include "waveblocks/innerproducts/gauss_hermite_qr.hpp"
#include "waveblocks/innerproducts/tensor_product_qr.hpp"
#include "waveblocks/propagators/Hagedorn.hpp"
#include "waveblocks/observables/energy.hpp"
#include "waveblocks/io/hdf5writer.hpp"
//namespace for convenience
using namespace waveblocks;
//Main function
int main()
{
//Simulation settings
const int N = 1;
const int D = 2;
const int K = 4;

const real_t sigma_x = 0.5;
const real_t sigma_y = 0.5;

const real_t tol = 1e-14;

const real_t T = 12;
const real_t dt = 0.01;

const real_t eps = 0.1;

using MultiIndex = wavepackets::shapes::TinyMultiIndex<unsigned long, D>;

//The parameter set of the initial wavepacket
CMatrix<D,D> Q = CMatrix<D,D>::Identity();
CMatrix<D,D> P = complex_t(0,1) * CMatrix<D,D>::Identity();
RVector<D> q = {-3.0, 0.0};
RVector<D> p = { 0.0, 0.5};
complex_t S = 0.0;
wavepackets::HaWpParamSet<D> param_set(q,p,Q,P,S);

//Basis shape
wavepackets::shapes::ShapeEnumerator<D, MultiIndex> enumerator;
wavepackets::shapes::ShapeEnum<D, MultiIndex> shape_enum = enumerator.generate(wavepackets::shapes::HyperCubicShape<D>(K));

//Gaussian Wavepacket phi_00 with c_00 = 1
Coefficients coeffs = Coefficients::Zero(std::pow(K, D), 1);
coeffs[0] = 1.0;
Coefficients coefforig = Coefficients(coeffs);

//Assemble packet
wavepackets::ScalarHaWp<D,MultiIndex> packet;
packet.eps() = eps;
packet.parameters() = param_set;
packet.shape() = std::make_shared<wavepackets::shapes::ShapeEnum<D,MultiIndex>>(shape_enum);
packet.coefficients() = coeffs;

//Defining the potential
typename CanonicalBasis<N,D>::potential_type potential = [sigma_x,sigma_y](CVector<D> x)
	{
    return 0.5*(sigma_x*x[0]*x[0] + sigma_y*x[1]*x[1]).real();
	};
	
typename ScalarLeadingLevel<D>::potential_type leading_level = potential;
typename ScalarLeadingLevel<D>::jacobian_type leading_jac = [sigma_x,sigma_y](CVector<D> x) 
    {
    return CVector<D>{sigma_x*x[0], sigma_y*x[1]};
    };
typename ScalarLeadingLevel<D>::hessian_type leading_hess = [sigma_x,sigma_y](CVector<D> /*x*/)
    {
    CMatrix<D,D> res;
    res(0,0) = sigma_x;
    res(1,1) = sigma_y;
    return res;
    };

ScalarMatrixPotential<D> V(potential,leading_level,leading_jac,leading_hess);

//Quadrature rules
using TQR = innerproducts::TensorProductQR<innerproducts::GaussHermiteQR<K+4>,innerproducts::GaussHermiteQR<K+4>>;

//Defining the propagator
propagators::Hagedorn<N,D,MultiIndex, TQR> propagator;

io::hdf5writer<D> writer("harmonic_2D_cpp.hdf5");
writer.set_write_energies(true);
writer.set_write_norm(true);
writer.prestructuring<MultiIndex>(packet,dt);

//Time loop(propagation)
for (real_t t = 0; t < T; t += dt) 
	{
    real_t ekin = observables::kinetic_energy<D,MultiIndex>(packet);
    real_t epot = observables::potential_energy<ScalarMatrixPotential<D>,D,MultiIndex,TQR>(packet,V);

	writer.store_packet(packet);
    writer.store_energies(epot,ekin);
    writer.store_norm(packet);
    	
    //Propagate
    propagator.propagate(packet,dt,V);

    //Assure constant coefficients
    auto diff = (packet.coefficients() - coefforig).array().abs();
    auto norm = diff.matrix().template lpNorm<Eigen::Infinity>();
    bool flag = norm > tol ? false : true;
    std::cout<<flag<<'\n';
    }
writer.poststructuring();
}
\end{lstlisting}
